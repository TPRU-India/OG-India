%!TEX root = ../OGUSAdoc.tex

The overlapping generations model is a workhorse of dynamic fiscal analysis. \ogindia is dynamic in that households in the model make consumption, savings, and labor supply decisions based on their expectations over their entire lifetime, not just the current period. Because \ogindia is a general equilibrium model, behavioral changes by households and firms can cause macroeconomic variables and prices to adjust.

But the main characteristic that differentiates the overlapping generations model from other dynamic general equilibrium models is its realistic modeling of the finite lifetimes of individuals and the cross-sectional age heterogeneity that exists in the economy. One can make a strong case that age heterogeneity and income heterogeneity are two of the main sources of diversity that explain much of the behavior in which we are interested for policy analysis.

\ogindia can be summarized as having the following characteristics.
\begin{itemize}
  \item Households
  \begin{itemize}
    \item overlapping generations of finitely lived households
    \item households are forward looking and see to maximize their expected lifetime utility, which is a function of consumption, labor supply, and bequests
    \item households choose consumption, savings, and labor supply every period.
    \item The only uncertainty households face is with respect to their mortality risk
    \item realistic demographics: mortality rates, fertility rates, immigration rates, population growth, and population distribution dynamics
    \item heterogeneous lifetime income groups within each age cohort, calibrated from U.S. tax data
    \item incorporation of detailed household tax data from \taxcalc microsimulation model
    \item calibrated intentional and unintentional bequests by households to surviving generations
  \end{itemize}
  \item Firms
  \begin{itemize}
    \item representative perfectly competitive firm maximizes static profits with general CES production function by choosing capital and labor demand
    \item exogenous productivity growth is labor augmenting technological change
    \item firms face a corporate income tax as well as various depreciation deductions and tax treatments
  \end{itemize}
  \item Government
  \begin{itemize}
    \item government collects tax revenue from households and firms
    \item government distributes transfers to households
    \item government spends resources on public goods
    \item government can run deficits and surpluses
    \item a stabilization rule (budget closure rule) must be implemented at some point in the time path if government debt is growing at a rate permanently different from GDP.
  \end{itemize}
  \item Aggregate, market clearing, and international
  \begin{itemize}
    \item Aggregate model is deterministic (no aggregate shocks)
    \item Three markets must clear: capital, labor, and goods markets
    \item
  \end{itemize}
\end{itemize}

%%% Put summary of the general incentives in the model, overall implications of the assumptions, and particularly how these interact with tax policy

We will update this document as more detail is added to the model.
